\documentclass[12pt]{article}

\usepackage[utf8]{inputenc}
%\usepackage[frenchb]{babel}
\usepackage[T1]{fontenc}
%\usepackage{t1enc}
\usepackage{longtable}
\usepackage{layout}
\usepackage[top=2.8cm, bottom=2.8cm, left=2.8cm, right=2.8cm]{geometry}
\usepackage{setspace}
\usepackage{graphicx}
\usepackage{libertine}
\usepackage{sidecap}
\usepackage{caption}
\usepackage{fancyhdr}
\usepackage{titlesec}
\usepackage{subfig}
\usepackage{placeins}
\usepackage{multirow}
\usepackage{listings}
\usepackage{dsfont}
\usepackage{amsmath}



\begin{document}
\title{Litterature review}
\maketitle

\section{Definition}

\paragraph{}
\textbf{Brightness Temperature} : The temperature at which a black body in thermal equilibrium with its surroundings would have to be in order to duplicate the observed specific intensity of an object at a frequency v. Often in radio astronomy brightness temperature (TB) is used as a measure of received intensity.

\vspace{0.5cm}

\textbf{Anvil cloud} : Anvil clouds, which are mostly composed of ice particles, form in the upper parts of thunderstorms. They get their anvil shape from the fact that the rising air in thunderstorms expands and spreads out as the air bumps up against the bottom of the stratosphere. This is because the air in the stratosphere is warmer than the rising air in the anvil, and so prevents the relatively cooler anvil air from rising any farther.

\vspace{0.5cm} 
\textbf{Equivalent reflectivity factor Z and dBZ}: Z expresses the volume of reflector per volume area (unit is mm\textsuperscript{6}.m\textsuperscript{-3}). dBZ is defined by $dBZ= 10.log_{10}(\frac{Z}{Z_{0}})$.

\vspace{0.5cm}
\textbf{Plan position indicator and Range height indicator} : These are the two main types of scans in meteorology. Plan position indicator refers to a scan with a constant elevation and a rotation of the azimuth angle. Range height indicator refers to a constant azimuth angle while the elevation angle varies.  

\vspace{0.5cm}
\textbf{Polarimetric radar/variable}:

\vspace{0.5cm}
\textbf{Standard reflectivity}:

\vspace{0.5cm}
\textbf{Doppler velocity radar parameters}:

\vspace{0.5cm}
\textbf{Energy dissipation rate (EDR)}
Turbulent flows consist of eddiyes of various size range. The kinetic energy cascades down from large to small eddies by interactionnal forces between the eddies. At very small scale, the energy of the eddies dissipates into heatdue to viscous forces. Energy dissipation rate is the parameter to determine the amount of energy lost by the viscous forces in the turbukent flow. 

\section{Litterature review}

\vspace{0.5cm}

\subsection{Articles focused on forests fuel study}

\vspace{0.5cm}


\begin{center}

\begin{longtable}{|p{4.5cm}|p{2cm}|p{0.8cm}|p{7.5cm}|}

\hline
 & & & \tabularnewline
\textbf{Title} & \textbf{Author} & \textbf{Date} & \textbf{Keypoints} \tabularnewline
 & & & \endfirsthead
 \hline
 
 \hline
 & & & \tabularnewline
\textbf{Title} & \textbf{Author} & \textbf{Date} & \textbf{Keypoints} \tabularnewline
 & & & \endhead
 \hline
 
 \hline
\endlastfoot
 \hline
 
 \hline
\endfoot
 \hline
 
\raggedright The line intersect method in forest fuel sampling & Van Wagner & 1968 & \raggedleft 
\begin{itemize} 
\item Simple and cost efficient method to get the volume or weight per area of woody fuel
\item Error become small with assumption of random orientation of the wood pieces
\end{itemize} 
\tabularnewline

\hline 


\raggedright Practical aspects of the line intersect method & Van Wagner & 1982 & \raggedleft 
\begin{itemize} 
\item Corretion factor for the ground slope
\item Importance of total line lenght and number of diameter classes used for the precision.
\end{itemize} 
\tabularnewline

\hline


\raggedright Development and structure of the Canadian Forest Fire Weather Index System Forestry Technical Report & Van Wagner& 1987 & \raggedleft 
\begin{itemize} 
\item A
\end{itemize} 
\tabularnewline

\hline

\raggedright Estimating plant biomass : a review of techniques & Catchpole& 1992 & \raggedleft 
\begin{itemize} 
\item A
\end{itemize} 
\tabularnewline

\hline



\raggedright Heat release rate : the single most important variable in fire hazard & V. Babrauskas & 1992 & \raggedleft 
\begin{itemize} 
\item A
\end{itemize} 
\tabularnewline



\hline
 
\raggedright Prescribed burning of thining slash in regrowth stands of farri & Mc Caw & 1997 & \raggedleft 
\begin{itemize} 
\item Consumption of woody fuels slash < 10cm is inversely related to the moisture content of the litter profile
\item Total amount of fuel consumed varies from 31 to 89 \%
\item Four woody fuel consumption of woody fuel exist. They are tested with an eucalyptus fire.
\item Gould and Cheney model assumes 50 \% will be consumed. Minimum error but doesn't take into account extreme events
\item CONSUME activity andSouthern woddy models underpredict observation
\item CONSUME Western woody model has little bias and good prediction
\item BURNUP model has poor performance with natural fuel, but improves for fuel from clearcut operations.
\end{itemize} 
\tabularnewline

\hline

\raggedright A new method to estimate fuel surface area-to-volume ration using water immersion & P.M. Fernandes& 1998 & \raggedleft 
\begin{itemize} 
\item A
\end{itemize} 
\tabularnewline

\hline


\raggedright Loss of carbon during controlled regeneration burns in Eucalyptus obliqua forest & A. Slijepcevic & 2001 & \raggedleft 
\begin{itemize} 
\item Sampling method for fine fuel (diameter < 2.5cm)
\item Line intercept method with triangles used for sampling of fuel > 2.5cm in diameter. 
\item Getting carbon mass from biomass (50\%)
\item fine fuels negligeable (~5 \% of the biomass) 
\item Around 70 \% of carbon released comes from fuel greater than 7 cm in diameter
\item Between 44 and 59 \% of carbon loss
\end{itemize} 
\tabularnewline

\hline

\raggedright Testing woody fuel consumption models for application in Australian southern eucalypt forest fires & J.J. Hollis & 2001 & \raggedleft 
\begin{itemize} 
\item Woody fuel consumption varies between 9.1 and 89.9 \% : large deviations
\item Fire  Management in Australian Forests states that 50\% of the total fuel load is a good figure for wildfires for certain range conditions
\end{itemize} 
\tabularnewline

\hline




\end{longtable}

\end{center}
\newpage
\subsection{Articles focused on Extreme fire behaviour}
\begin{center}
\begin{longtable}{|p{4.5cm}|p{2cm}|p{0.8cm}|p{7.5cm}|}

\hline
 & & & \tabularnewline
\textbf{Title} & \textbf{Author} & \textbf{Date} & \textbf{Keypoints} \tabularnewline
 & & & \endfirsthead
 \hline
 
 \hline
 & & & \tabularnewline
\textbf{Title} & \textbf{Author} & \textbf{Date} & \textbf{Keypoints} \tabularnewline
 & & & \endhead
 \hline
 
 \hline
\endlastfoot
 \hline
 
 \hline
\endfoot
 \hline
 
\raggedright Unusual phenomena in an extreme busfire & J. Dold & 2005 & \raggedleft 
\begin{itemize} 
\item A
\end{itemize} 
\tabularnewline

\hline

\raggedright Firebrands and spotting ignition in large-scale fires & E. Koo & 2010 & \raggedleft 
\begin{itemize} 
\item A
\end{itemize} 
\tabularnewline

\hline
 
\raggedright Fire whirls due to surrounding flame sources and the influence of the rotation speed on flame height & R. Zhou & 2007 & \raggedleft 
\begin{itemize} 
\item A
\end{itemize} 
\tabularnewline

\hline
 
\raggedright Photographs and analysis of an unusually large and longlived firewhirl & M. Umsheid & 2006 & \raggedleft 
\begin{itemize} 
\item Unusually long lived fire whirl lasted 20 minutes, and towered at 200m.
\item Firewhirl not associated with strong reflectivity (20, 30 dBZ)
\end{itemize} 
\tabularnewline

\hline

\raggedright Atmospheric interactions with wildland fire
behaviour – I,II & E. Potter & 2012 & \raggedleft 
\begin{itemize} 
\item A
\end{itemize} 
\tabularnewline

\hline





\end{longtable}
\end{center}


\newpage 

\subsection{Articles focused on Pyrocumulonimbus (PyroCb for short)}
\begin{center}
\begin{longtable}{|p{4.5cm}|p{2cm}|p{0.8cm}|p{7.5cm}|}

\hline
 & & & \tabularnewline
\textbf{Title} & \textbf{Author} & \textbf{Date} & \textbf{Keypoints} \tabularnewline
 & & & \endfirsthead
 \hline
 
 \hline
 & & & \tabularnewline
\textbf{Title} & \textbf{Author} & \textbf{Date} & \textbf{Keypoints} \tabularnewline
 & & & \endhead
 \hline
 
 \hline
\endlastfoot
 \hline
 
 \hline
\endfoot
 \hline
 


\raggedright The Chisholm firestorm : observed microstructure, precipitation and lightning activity & D. Rosenfield & 2007 & \raggedleft 
\begin{itemize} 
\item High convective effect leads to injection of smoke in the stratosphere
\item Pyro-Cb are formed of very small drops that are slow to coalesce into rain drops due to large concentration of CCN from the fire smoke and/or the updraft velocity.
\item PyroCb delay rain collaspe and generate updrafts for a longer period, then are favorable to increase fire severity and spotting.
\item Cloud top temperature can be estimated equal to the thermal brightness temperature under certain conditions (low brightness temp differential between 10. and 12 ym)
\end{itemize} 
\tabularnewline

\hline

\raggedright Violent pyro-convective storm devastates Australia's capital and pollutes the atmosphere & M.Fromm & 2006 & \raggedleft 
\begin{itemize}
\item Radar data of the pyroCb
\item PyroCb generate an anvil (hard to observe with satellite) 
\item The aerosols in the pyro-Cb will asbord/intercept solar radiation once in the stratosphere
\item Damage consistend with a tornadic event rather than fire whirl for three reasons:
	\begin{itemize}
	\item Path dimensions (20km lomg and 450 wide)
	\item Break in the damage path (temporarily lift of the vortex: therefore it wouldn't touche the ground)
	\item Damage extent beyond the burn zone
	\end{itemize}
\item 1dBZ radar contour allows to see the pyroCb can give a good 3D repreentation
\item Fire affects the atmosphere :
	\begin{itemize}
	\item Disturbs the airflow
	\item Injects smoke/heat/emittants
	\end{itemize}	 
\item Atmosphere conditions affect fire: wind/moisture/temperature influence fuel dryness and spotting, control fire behaviour and rate of spread. 

	
\end{itemize} 
\tabularnewline

\hline

\raggedright Strastospheric impact of the Chisholm pyrocumulonimbus eruption : 1 and 2 & M.Fromm & 2008 & \raggedleft 
\begin{itemize} 
\item Use of satellite with views at different slants and sprectral width (MISR) to detect thin clouds, elevated aerosol layers and near surface plume. Wind correction applied.
\item TOMS Aerosol Index (AI) to detect UV absorbing aerosols located in the UTLS.
\item Between 0.3 and 2.2 \% of the fuel consumed during pyroconvection ended up in the stratosphere
\item 
\item Sensoring the pyroCb with LIDAR data
\end{itemize} 
\tabularnewline

\hline

\raggedright The untold Story of Pyrocumulonimbus & M.Fromm & 2010 & \raggedleft 
\begin{itemize} 
\item Average cloud-top altitude and pressure of a pyroCb are 11.6km and 223hPa
\end{itemize} 
\tabularnewline

\hline

\raggedright Nuclear Winter: global consequences of multiple nuclear explosions & R. Turco & 1983 & \raggedleft 
\begin{itemize} 
\item A
\item 
\end{itemize} 
\tabularnewline

\hline

\raggedright The 2013 rim fire: Implications for Predicting Extreme Fire Spread, Pyroconvection, and Smoke Emissions & D. Peterson & 2015 & \raggedleft 
\begin{itemize} 
\item Radiant heat can be used as an approximation for the fire intensity
\item FRP was maximum in late evening and minimum in the morning
\item Meteorological conditions: 
	\begin{itemize}
	\item Long and short term drought
	\item Dry fuel
	\item Low relative humidity (<40\%)
	\item Dry lower troposhere
	\item Upper-level disturbance near the fire associated with critical FRP
	\end{itemize}
\item Low atmospheric fire weather are unable to identify the most extreme fire behaviour
\item Fire spread/intensity and altitude smoke injection are related but not predicted by the same variables
\item Good precursor for pyro Cb : 
	\begin{itemize}
	\item Large and intense fire
	\item Ambient mid-level moisture
	\item upper level instability
	\end{itemize}
\end{itemize} 
\tabularnewline

\hline
\raggedright Severe convective storms initiated by intense wildfires: Numerical simulations of pyro-convection and pyro-tornadogenesis & P.Cunningham & 2009 & \raggedleft 
\begin{itemize} 
\item A
\end{itemize} 
\tabularnewline

\hline

\raggedright An Australian pyro-tornadogenesis event & R. McRae & 2012 & \raggedleft 
\begin{itemize} 
\item Tornado definition (by Glickman,2000): 
a violently rotating column of air, in contact with the surface, pendant from a cumuliform cloud, and often (but not always) visible as a funnel cloud.
\item Path of the tornado shows stronger reflectivity (in this case the radar beam was under the pyroCb)
\item Movement of the tornado is along the centerline of the fire plume. (support the idea that the tornado was indeed a pyrogenic event)
\end{itemize} 
\tabularnewline

\hline

\raggedright Smoke-Column Observations from Two Forest Fires Using Doppler Lidar and Doppler Radar  & R. Banta & 1992 & \raggedleft 
\begin{itemize} 
\item Use of a 3.2cm-wavelength radar.
\item Detection of horizontal vortices with vertical motion on the edges of the plume. Cause is strong flow at the edge and weak flow in the center of the plume.
\item Depolarization ration measurement used to get information on the shape of the object in the plume.
\item Criterion for wind-driven vs buyoancy-driven fire: ratio between kinetic energy (KE) of the buyoancy and KE of the ambient wind flow. 
\item Rotation contributes to a stronger fire intensity by increasing the updraft.
\item Turbulent dissipations are limited by helicity.

\end{itemize} 
\tabularnewline

\hline

\end{longtable}
\end{center}
\newpage

\subsection{Articles focused on Vorticity/Turbulence in plumes}

\begin{center}
\begin{longtable}{|p{4.5cm}|p{2cm}|p{0.8cm}|p{7.5cm}|}

\hline
 & & & \tabularnewline
\textbf{Title} & \textbf{Author} & \textbf{Date} & \textbf{Keypoints} \tabularnewline
 & & & \endfirsthead
 \hline
 
 \hline
 & & & \tabularnewline
\textbf{Title} & \textbf{Author} & \textbf{Date} & \textbf{Keypoints} \tabularnewline
 & & & \endhead
 \hline
 
 \hline
\endlastfoot
 \hline
 
 \hline
\endfoot
 \hline
 
 
\raggedright Note of the observation of small-scale amospheric turbulences by Doppler radar techniques & R. Lhermitte & 1969 & \raggedleft 
\begin{itemize} 
\item A
\end{itemize} 
\tabularnewline

\hline 

\raggedright Fire whirls, why, when and where & Countryman & 1971 & \raggedleft 
\begin{itemize} 
\item Fire whirl form from the rising of unstable air, and are linked to the heated surface.
\item They don't lift off along their path. 
\item Fire whirld wind speeds may reach and exceed 300 miles/h.
\end{itemize} 
\tabularnewline

\hline

\raggedright Intense atmospheric vortices associated with a 1000 MW fire & C. Church & 1980 & \raggedleft 
\begin{itemize} 
\item A
\end{itemize} 
\tabularnewline

\hline

\raggedright Radar detection of turbulence in precipitation environments & A. Bohne & 1982 & \raggedleft 
\begin{itemize} 
\item A
\end{itemize} 
\tabularnewline

\hline

\raggedright Streamwise Vorticity: The Origin of Updraft Rotation in Supercell Storms  & R. Davies-Jones & 1984 & \raggedleft 
\begin{itemize} 
\item A
\end{itemize} 
\tabularnewline

\hline 


\raggedright Development of large vortices on prescribed fires & D McRae & 1990 & \raggedleft 
\begin{itemize} 
\item Fire whirl usually associated with low winds speed (< 10 m/s) and therefore nearly vertical smoke column		
\item Production of intense firewirls is believed to be a function of the interation of the rate of energy release and the degree of atmospheric instability in the 1000 to 3000 m layer
\item Fire whirls need available vorticity close to form
\item Two types of fire whirl observed: 
	\begin{itemize}
	\item Type 1: Whirlinds formed in pair in the leeward side of the convective column. Vorticity is tilted to match the vertical motion field of the column
	\item Type 2: Entire column goes in rotation and lead to large whirls. Cause is likely to be stretching of existing vertical vorticity adveted from the surroundings or result of buoyancy.
	\end{itemize}
\end{itemize} 
\tabularnewline

\hline


\raggedright NEXRAD Detection of Hazardous turbulence & J. Williams & 2006 & \raggedleft 
\begin{itemize} 
\item A
\end{itemize} 
\tabularnewline

\hline


\raggedright Turbulent Plumes in Nature & A. Woods & 2010 & \raggedleft 
\begin{itemize} 
\item A
\end{itemize} 
\tabularnewline

\hline



\raggedright Review of vortices in wildland fire & J Forthofer & 2011 & \raggedleft 
\begin{itemize} 
\item Fire whirls characterized by sudden formation, erratic movement, often sudden dissipation. Definition: vertically oriented, intensely rotating column of gas found in or near fires.
\item Contribution to the formation of firewhirls:
	\begin{itemize}
	\item Lee slope
	\item Superadiabatic or dry lapse rate in the atmoshere (big decrease of temerature with height)
	\item Vorticity associated with passage of a cold front
	\end{itemize}
\item Range varies between 10m-3km and 10m/s-100m/s for the wind velocity
\item Can be strong enough to move over 1000m away from the fire and lift a house
\item Formation of a firewhirl requires ambient vorticity and a concentrating mechanism (buoyancy flow from the fire)
\item Reduction by one order of magnitude of the tubulent mixing (due to a balance between radial gradient pressure with the low pressure in the core and centrifuge acceleration), increasing flame lenght and burning rate.
\item cyclostrophic flow reached above the ground. Near the ground, drag force cut the cyclostrophic balance, the main flow is then due to the pressure gradient and fuel/gas are draggged toward the core and then lifted.
\end{itemize} 
\tabularnewline

\hline

\end{longtable}
\end{center}
 
\newpage 
 
\subsection{Articles focused radar}
\begin{center}
\begin{longtable}{|p{4.5cm}|p{2cm}|p{0.8cm}|p{7.5cm}|}

\hline
 & & & \tabularnewline
\textbf{Title} & \textbf{Author} & \textbf{Date} & \textbf{Keypoints} \tabularnewline
 & & & \endfirsthead
 \hline
 
 \hline
 & & & \tabularnewline
\textbf{Title} & \textbf{Author} & \textbf{Date} & \textbf{Keypoints} \tabularnewline
 & & & \endhead
 \hline
 
 \hline
\endlastfoot
 \hline
 
 \hline
\endfoot
 \hline
 

\raggedright A real time four dimensional Doppler dealiasing Scheme & C. James & 2001 & \raggedleft 
\begin{itemize} 
\item A
\end{itemize} 
\tabularnewline

\hline



\raggedright A Mobile Rapid-Scanning X-band Polarimetric (RaXPol) Doppler Radar System & A. Pazmani & 2012 & \raggedleft 
\begin{itemize} 
\item A
\end{itemize} 
\tabularnewline

\hline

\raggedright Applications in Low-Power Phased Array Weather
Radars & R. Palumbo & 2016 & \raggedleft 
\begin{itemize} 
\item A
\end{itemize} 
\tabularnewline

\hline






\end{longtable}
\end{center}

\end{document}
